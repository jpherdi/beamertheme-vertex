
% multi language typesetting package
\usepackage{polyglossia}
\setmainlanguage{english}

% quoting, \enquote etc...
\usepackage{csquotes}    

% labels and references, bookmark adds functionality
\usepackage{hyperref}
\usepackage{bookmark}

% Package hyperref, float, listings improvements
\usepackage{scrhack}

% math-commands
\usepackage{amsmath}
% math-symbols
\usepackage{amssymb}
% additional math-commands
\usepackage{mathtools}

% unicode math
\usepackage[
  mathrm=sym,
  math-style=ISO,    % ┐
  bold-style=ISO,    % │
  sans-style=italic, % │ ISO-standard 
  nabla=upright,     % │
  partial=upright,   % ┘
  warnings-off={           % ┐
    mathtools-colon,       % │ switch off unnecessary warning
    mathtools-overbracket, % │
  },                       % ┘
]{unicode-math}

% Zahlen und Einheiten
\usepackage[
  locale=UK,                   % english settings
  separate-uncertainty=true,   % errors always with \pm
  per-mode=symbol-or-fraction, % / in inline math, fraction in display math
]{siunitx}

% nice fractions inline
\usepackage{xfrac}

% Dirac notation brackets
\usepackage{braket}

% better captions
\usepackage[
  labelfont=bf,        % Table x: Figure y: bold
  font=small,          % font little smaller than in document
  width=0.7\textwidth, % smaller width of caption
]{caption}

% subfigure, subtable, subref
\usepackage{subcaption}

% graphics can be used
\usepackage{graphicx}

% better tables
\usepackage{booktabs}

% improvement on typesetting
\usepackage{microtype}

% For extended footnote functionality
% \usepackage{footmisc}

% For citations
\usepackage[
  % style=numeric,
  % style=authoryear-ibid,
  % style=verbose,
  doi=false,
  style=phys,
  % sorting=none,
  backend=biber,
  autolang=hyphen,
  articletitle=false,
  chaptertitle=false,
  biblabel=brackets,
  eprint=false,
  pageranges=false,
  url=false,
  % useauthor=false,
]{biblatex}
\addbibresource{lit.bib}

% draw with code
\usepackage{tikz}

% moveability with absolute text positions, great for beamer
\usepackage[overlay]{textpos}

% additional functionality of \includegraphics
\usepackage[export]{adjustbox}

% wrapfigures more for text
\usepackage{wrapfig}

% Trennung von Wörtern mit Strichen
\usepackage[shortcuts]{extdash}

% multicolumns
\usepackage{multicol}

\linespread{1.1}

\makeatletter
% % Default:
% \def\@makefnmark{\hbox{\@textsuperscript{\normalfont\@thefnmark}}}
\renewcommand{\@makefnmark}{\makebox{\normalfont[\@thefnmark]}}
\makeatother

\usetheme{vertex}